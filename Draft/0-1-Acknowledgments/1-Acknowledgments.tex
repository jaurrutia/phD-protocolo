% !TeX root = ../tesis.tex

\begin{acknowledgements}
\addcontentsline{toc}{chapter}{\protect\numberline{}Acknowledgements}
\vfill
En primera instancia, agradezco a mi tutor, el Dr. Alejandro Reyes Coronado, por su constante seguimiento y crítica a mis actividades, así como a la Dra. Citlali Sánchez Aké y al Dr. Giuseppe Pirruccio ---mi Comité Tutor---,  por su acompañamiento en procesos medulares durante el desarrollo de esta tesis de maestría. Asimismo, doy las gracias a la Dra. Svetlana Mansurova y al Dr. Rubén Ramos García, por ser unos de los detonantes del tema de investigación que se estudia en este trabajo. De igual manera, ofrezco mi agradecimiento a los docentes del Posgrado en Ciencias Físicas (PCF), con quien me formé durante mis estudios de maestría, por su entusiasmo y compromiso en sus cátedras pero, sobre todo, por haber reinventado su método de enseñanza durante el periodo de  distanciamiento social debido a la pandemia del SARS-COV-2. Finalmente, agradezco a la Dra. Alma Karen González Alcalde, por ser un ejemplo a seguir en todo ámbito.

Agradezco a la Universidad Nacional Autónoma de México (UNAM), y a su Facultad de Ciencias (FC), por abrirme una vez más sus puertas, así como al Consejo Nacional de Ciencia y Tecnología (CONACyT)  por otorgarme la beca 1044708, que me permitió dedicarme al programa de maestría del PCF de tiempo completo durante dos años. Al PCF, le ofrezco mi agradecimiento por haberme considerado dos veces en su Programa de Apoyo a los Estudios de Posgrado (PAEP) al proporcionarme el equipo de cómputo con el que se escribió esta tesis y el acceso a una licencia de COMSOL Multiphysics\texttrademark{}. Finalmente, doy gracias  al proyecto de investigación DGAPA-UNAM PAPIIT IN107122 por brindarme el uso del equipo de cómputo donde se realizaron los cálculos presentados en este trabajo.

En términos personales doy gracias a mi mamá, Irma AG, por su siempre creciente afecto y confianza; a mi papá, Humber US, por su sincera escucha; a mis hermanas, Abby UA y Diana UA, por dejar de lado nuestros disgustos cuando más lo necesitamos; y a Clau GR, por inspirarme y por llevarme a aventuras cada vez más vehementes. También, agradezco al Dr. Fermín VH por darme la oportunidad de conocerlo fuera y dentro de las aulas ---virtuales, principalmente--- de la FC y por demostrar siempre un ahínco contagioso por la enseñanza. De la misma manera, ofrezco mi agradecimiento a las amistades, nuevas y viejas, que me acompañaron durante este tiempo: Adrián AJ, Eduardo VA, Isabel RM, Jesús CF, Jorge BD, Luis BG,  Leonardo GP y Yara MC. Por último, agradezco al perrito blanco que me cuidó por catorce años y a la perrita ámbar que continua con esa tarea.

\end{acknowledgements}
