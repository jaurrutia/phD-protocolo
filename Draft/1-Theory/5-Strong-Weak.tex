% !TeX root = ../tesis.tex

The scattering light problem addressed in Chapter \ref{ch:OpticalProperties} corresponds to a steady-state problem in a domain $\Omega$ whose optical properties are characterized by its electric permittivity $\varepsilon$ and magnetic permeability $\mu$, which in general are complex functions of the angular frequency $\omega$. To obtain the strong formulation of the light scattering problem, let us assume harmonic time dependent electric $\mathbfcal{E}(\vb{r},t)$ and magnetic fields $\mathbfcal{H}(\vb{r},t)$:
%
\begin{align}
    \mathbfcal{E}(\vb{r},t) = \vb{E}(\vb{r})\exp(-i\omega t)
        \qquad
        \text{and}
        \qquad
    \mathbfcal{H}(\vb{r},t) =  \vb{H}(\vb{r}) \exp(-i\omega t),
\end{align}
%
with $ \vb{E}(\vb{r})$ and $ \vb{H}(\vb{r})$ complex valued vectors. For harmonic time dependent EM fields in optical linear media, the Maxwell's equation are rewritten as \cite{jackson_classical_1999}
%
\index{Maxwell Equations!Harmonic Time Dependent}%
%
\begin{subequations}
    \label{eq:MaxwellsEq}
\begin{align}
    \nabla \cdot \qty(\varepsilon \vb{E})  &= \rho_\text{ext},  &\mbox{(Electric Gauss's law)},\\
    \nabla \cdot   \vb{B} &= 0, & \mbox{(Magnetic Gauss's law)},
            \label{eq:GaussM}\\
    \nabla \times \vb{E}  &= i\omega \mu \vb{H},  &  \mbox{(Faraday-Lenz's law)},\\
    \nabla \times \qty(\mu \vb{H})  &= \vb{J} - i\omega \varepsilon \vb{E}, & \mbox{(Ampère-Maxwell's law)},
\end{align}
\end{subequations}
%
where  $\rho_\text{ext}$ is the external volumetric charge density, $\vb{J}$ is the volumetric current density, and  $\vb{B} = \mu \vb{H}$. The volumetric current density is given by \cite{jackson_classical_1999}
%
\begin{align}
    \vb{J}(\vb{r},t) = \vb{J}_\text{ext}(\vb{r})\exp(-i\omega t) + \sigma \vb{E}(\vb{r})\exp(-i\omega t),
    \label{eq:ohmic}
\end{align}
%
where $\vb{J}_\text{ext}$ corresponds to the external current density, $\sigma$ is the frequency dependent conductivity of the domain $\Omega$ and the term $\sigma \vb{E}(\vb{r})\exp(-i\omega t)$ corresponds to the induced current density that arises from an Ohmic response of the domain $\Omega$.

The Maxwell's equations can be decouple to avoid working with two unknown functions, yielding \cite{jackson_classical_1999,jin_theory_2010,larson_finite_2013,bondeson_computational_2005}
%
\index{Helmholtz!Equation!Vectorial}%
%
\begin{align}
    \nabla \times \qty[\mu^{-1} \nabla \times \vb{E}] -(i\omega\sigma + \omega^2 \varepsilon) \vb{E} = i\omega \vb{J}_\text{ext},
    \label{eq:curlcurlE}
\end{align}
%
which is another formulation of the vectorial Helmholtz equation [Eq. \eqref{eq:Helmholtz}]. In the absence of an external current and charge densities, Eq. \eqref{eq:curlcurlE} becomes the Maxwell eigenvalue problem \cite{larson_finite_2013}, which corresponds to the strong formulation of the light scattering problem alongside the  boundary conditions stated by Maxwell's equations [Eqs. \eqref{eq:MaxwellsEq}]. Therefore, the strong formulation of the light scattering problem is given by \cite{larson_finite_2013,jin_theory_2010,bondeson_computational_2005}
%
\index{Partial Differential Equation (PDE)!Maxwell Eigenvalue Problem!Strong Formulation}%
\index{Scattering!Problem!Strong Formulation}%
\index{Boundary Conditions!Maxwell Eigenvalue Problem}%
%
\begin{subequations}
    \label{eq:Scatt-Strong-All}
\begin{tcolorbox}[title = Maxwell's Eigenvalue Problem and Boundary Conditions (Strong Formulation), ams align, breakable ]
    \nabla \times \qty[\mu^{-1} \nabla \times \vb{E}] - \kappa^2 \vb{E} &= \vb{0},
        &
        \text{with}
        \qquad
        \kappa^2 = (i\omega\sigma +\omega^2 \varepsilon),
    \label{eq:MaxwellEigenvalue}
    \\
    \vu{n}\times \vb{E}(\vb{r})\eval_{\partial \Omega} &= \vb{E}_\text{D},
        &\mbox{(Dirichlet Boundary Condition)},
    \label{eq:Dirichlet}
    \\
    \mu^{-1} \nabla \times \vb{E} \times \vu{n} \eval_{\partial \Omega} &=  \vb{E}_\text{N},
        &\mbox{(Neumann Boundary Condition)},
    \label{eq:Neumann}
\end{tcolorbox}
\end{subequations}
%
\noindent
with $\partial\Omega$ the boundary of $\Omega$ and $\vu{n}$ a normal vector to $\partial\Omega$. A Dirichlet type boundary condition is described in Eq. \eqref{eq:Dirichlet} while the Eq. \eqref{eq:Neumann} corresponds to Neumann type boundary condition. Both of such boundary conditions are  equivalent to Eq. \eqref{eq:contuinity} in Section \ref{ss:Fields} when  $\vb{E}_\text{D} =  \vb{E}_\text{N} = \vb{0}$.

To build the  weak formulation of the light scattering problem let us choose a set of $N$ linearly independent vectorial interpolating functions $\{\boldsymbol{\eta}_j\}_{j\leq N}$. If the Eq. \eqref{eq:MaxwellEigenvalue} is projected onto one interpolating function $\boldsymbol{\eta}_j$ and the result is integrated in the domain $\Omega$, one obtains
%
\begin{align}
    \int_\Omega \boldsymbol{\eta}_j\cdot \nabla \times \qty[\mu^{-1} \nabla \times \vb{E}] \dd{\Omega}
        = \int_\Omega \kappa^2 \boldsymbol{\eta}_j \cdot   \vb{E}  \dd{\Omega}.
    \label{eq:int-Scatt}
\end{align}
%
The left-hand side of Eq. \eqref{eq:int-Scatt} can be simplified by employing the vectorial identity $\nabla\cdot(\vb{A}\times \vb{B}) = \vb{B}\cdot\nabla\times\vb{A}-\vb{A}\cdot\nabla\times\vb{B}$, with $\vb{A} = \boldsymbol{\eta}_j$ and $\vb{B} = \mu^{-1}\nabla\times\vb{E}$, and performing a one-fold integration by parts, yielding
%
\begin{align}
    \int_\Omega \boldsymbol{\eta}_j\cdot \nabla \times \qty[\mu^{-1} \nabla \times \vb{E}] \dd{\Omega}
        =&  \int_\Omega \qty(\mu^{-1}\nabla\times\vb{E})\cdot \qty(\nabla\times\boldsymbol{\eta}_j)\dd{\Omega}
            \notag \\
        & - \oint_{\partial\Omega}\qty[\boldsymbol{\eta}_j\times\qty(\mu^{-1}\nabla\times\vb{E})]\cdot \vu{n}\dd{(\partial\Omega)},
    \label{eq:IntParts}
\end{align}
%
where the last term in the right-hand side is obtained by the Gauss's Theorem. Substituting Eq. \eqref{eq:IntParts} into Eq. \eqref{eq:int-Scatt} and applying the boundary condition given by Eq. \eqref{eq:Neumann}, the Maxwell's eigenvalue problem in its weak formulation is obtained:
%
\index{Partial Differential Equation (PDE)!Maxwell Eigenvalue Problem!Weak Formulation}%
\index{Scattering!Problem!Weak Formulation}%
%
\begin{subequations}
    \label{eq:Scatt-Weak-All}
\begin{tcolorbox}[title = Maxwell's Eigenvalue Problem and Boundary Conditions (Weak Formulation), ams align, breakable ]
    \int_\Omega \left\{\qty(\mu^{-1}\nabla\times\vb{E})\cdot \qty(\nabla\times\boldsymbol{\eta}_j) -  \kappa^2  \cdot   \vb{E} \cdot \boldsymbol{\eta}_j \right\} \dd{\Omega} & - \oint_{\partial\Omega} \qty(\boldsymbol{\eta}_j\times \vb{E}_\text{N})  \cdot\vu{n}\dd{(\partial\Omega)} = 0,
    \label{eq:MaxwellEigenvalue-weak}
    \\
    \vu{n}\times \vb{E}(\vb{r})\eval_{\partial \Omega} &= \vb{E}_\text{D}.
    \label{eq:Dirichlet-weak}
\end{tcolorbox}
\end{subequations}
%

The weak formulation problem of light scattering  [Eq. \eqref{eq:MaxwellEigenvalue-weak}] in the finite element approximation is reduced to the algebraic system of equations given by the Galerkin method in each of the $M$ subdomains $\Omega_k$ of $\Omega$, given by \cite{larson_finite_2013,jin_theory_2010}
%
\begin{subequations}
        \label{eq:Vec-FEM}
\begin{align}
    \mathbb{A} \vb{e} = \vb{0},
\end{align}%
with
\begin{align}
    A_{ij}  =     \int_{\Omega_k} \left\{\qty(\mu^{-1}\nabla\times\boldsymbol{\eta}_i)\cdot \qty(\nabla\times\boldsymbol{\eta}_j) -  \kappa^2  \cdot   \boldsymbol{\eta}_i \cdot \boldsymbol{\eta}_j \right\} \dd{\Omega_k}
            -& \oint_{\partial\Omega} \qty(\boldsymbol{\eta}_j\times \vb{E}_\text{N}) \cdot\vu{n}\dd{(\partial\Omega)},
\end{align}%
\end{subequations}%
\noindent%
%
where $A_{ij}$ are the matrix elements of $\mathbb{A}$ and  $\vb{e}$ is the vector containing the $N$ coefficients of the linear combination of the interpolating function $\boldsymbol{\eta}_j$ that results into an approximated solution $\tilde{\vb{E}}(\vb{r}) = \sum_i e_i \boldsymbol{\eta}_j(\vb{r})$ to the electric field. The choice of the coefficients $e_i$, its physical interpretation, and the set of the vectorial interpolating functions  is discussed in the next section.
