\tdplotsetmaincoords{60}{110}
 \pgfmathsetmacro{\rvec}{1. 3}
 \pgfmathsetmacro{\thetavec}{30}
 \pgfmathsetmacro{\varphivec}{60}
\begin{tikzpicture}[scale=3.5,tdplot_main_coords]
 %draw the NP
 %\draw[tdplot_screen_coords,ball color=yellow, opacity = 1] (0,0,0) circle (.05);
 %\draw[tdplot_screen_coords, color=yellow, opacity = 1] (0,0,0) circle (.05);

 \pgfmathsetseed{3}
     \draw[tdplot_screen_coords, ball color=yellow, opacity = 1,scale =.075]
     plot [smooth cycle, samples=8,domain={1:8}]
         (\x*360/8+5*rnd:0.5cm+1cm*rnd) node at (0,0) {};
 \pgfmathsetseed{3}
     \draw[tdplot_screen_coords, color=yellow, opacity = 1,scale =.075]
      plot [smooth cycle, samples=8,domain={1:8}]
      (\x*360/8+5*rnd:0.5cm+1cm*rnd) node at (0,0) {};


 % Set up some coordinates
  \coordinate (O) at (0,0,0);

 %determine a coordinate (P) using (r,\theta,\varphi) coordinates.   This command
 %also determines (Pxy), (Pxz), and (Pyz): the xy-, xz-, and yz-projections
 %of the point (P).
 %syntax: \tdplotsetcoord{Coordinate name without parentheses}{r}{\theta}{\varphi}
 \tdplotsetcoord{P}{\rvec}{\thetavec}{\varphivec}

 %draw figure contents
 %--------------------
 %draw the main coordinate system axes
     \draw[thick,- latex] (0,0,0) -- (1. 5,0,0) node[anchor=north east]{$x$};
     \draw[thick,- latex] (0,0,0) -- (0,1. 5,0) node[anchor=north west]{$y$};
     \draw[thick,- latex] (0,0,0) -- (0,0,1. 5) node[anchor=south]{$z$};

 %draw the main cartesian vector system
     \draw[thick,- latex, blue] (0,0,0) -- (1,0,0) node[anchor= south east]{$\vu{e}_x$};
     \draw[thick,- latex, blue] (0,0,0) -- (0,1,0) node[anchor=north west]{$\vu{e}_y$};
     \draw[thick,- latex, blue] (0,0,0) -- (0,0,1) node[anchor= east]{$\vu{e}_z$};

 %draw a vector from origin to point (P)
     \draw[thick,color=green, - latex] (O) -- (P);
     \node at (1,. 5,1. 1) {\color{green} $\vb{r}$};

 %draw projection on xy plane, and a connecting line
     \draw[dashed, color=green] (O) -- (Pxy);
     \draw[dashed, color=green] (P) -- (Pxy);
     \fill[green, opacity = .3] (O) --(Pxy)-- (P)--(O);
     \draw[- latex, tdplot_screen_coords,green](.42,.2)--(.8,.2);
     \node[tdplot_screen_coords] at (1.2,.2) {\color{green}\small Scattering plane};


 %draw the angle \varphi, and label it
     %syntax: \tdplotdrawarc[coordinate frame, draw options]{center point}{r}{angle}{label options}{label}
     \tdplotdrawarc[- latex]{(O)}{0. 5}{0}{\varphivec}{anchor=south}{$\varphi$}


 %set the rotated coordinate system so the x'-y' plane lies within the
     %"theta plane" of the main coordinate system
     %syntax: \tdplotsetthetaplanecoords{\varphi}
     \tdplotsetthetaplanecoords{\varphivec}

 %draw theta arc and label, using rotated coordinate system
     \tdplotdrawarc[tdplot_rotated_coords, - latex]{(0,0,0)}{0. 45}{0}{\thetavec}{anchor=north}{$\theta$}

 %draw some dashed arcs, demonstrating direct arc drawing
     \draw[dashed,tdplot_rotated_coords] (\rvec,0,0) arc (0:90:\rvec);
     \draw[dashed] (\rvec,0,0) arc (0:90:\rvec);

 %set the rotated coordinate definition within display using a translation
 %coordinate and Euler angles in the "z(\alpha)y(\beta)z(\gamma)" euler rotation convention
 %syntax: \tdplotsetrotatedcoords{\alpha}{\beta}{\gamma}
     \tdplotsetrotatedcoords{\varphivec}{\thetavec}{0}

 %translate the rotated coordinate system
 %syntax: \tdplotsetrotatedcoordsorigin{point}
     \tdplotsetrotatedcoordsorigin{(P)}

 %use the tdplot_rotated_coords style to work in the rotated, translated coordinate frame
     \draw[thick,tdplot_rotated_coords,- latex, purple] (0,0,0) -- (. 3,0,0) node[anchor=north west]{{\color{black}$\vu{e}_\theta,$}$\vu{e}_{\parallel}^\text{sca}$};
     \draw[thick,tdplot_rotated_coords,- latex,black] (0,0,0) -- (0,. 3,0) node[anchor=west]{$\vu{e}_\varphi$};
     \draw[thick,tdplot_rotated_coords,- latex,purple] (0,0,0) -- (0,-. 3,0) node[anchor= north west]{$\vu{e}_{\perp}^\text{sca}$};
     \draw[thick,tdplot_rotated_coords,- latex] (0,0,0) -- (0,0,. 3) node[anchor=south]{$\vu{k}^\text{sca}, \vu{e}_r$ };

 %set the rotated coordinate definition within display using a translation
 %coordinate and Euler angles in the "z(\alpha)y(\beta)z(\gamma)" euler rotation convention
 %syntax: \tdplotsetrotatedcoords{\alpha}{\beta}{\gamma}
     \tdplotsetrotatedcoords{\varphivec}{0}{0}

 %translate the rotated coordinate system
 %syntax: \tdplotsetrotatedcoordsorigin{point}
     \tdplotsetrotatedcoordsorigin{(Pxy)}

     \draw[thick,tdplot_rotated_coords,- latex, purple] (0,0,0) -- (. 3,0,0) node[anchor= west]{$\vu{e}_{\parallel}^\text{i}$};
     \draw[thick,tdplot_rotated_coords,- latex, blue] (0,0,0) -- (0,0,. 3) node[anchor= west]{$\vu{e}_z$};
     \draw[thick,tdplot_rotated_coords,- latex, purple] (0,0,0) -- (0,-. 3,0) node[anchor= north west]{$\vu{e}_{\perp}^\text{i}$};

 % plane wave
     \foreach \i in {-5}{%{-7,...,-2}{
         \draw[thick,tdplot_screen_coords,red, - latex] (\i/10,0,0)--(\i/10,1,0);}
     \node[tdplot_screen_coords] at (-4.75/10,1.1,0){\color{red}$\vb{k}^\text{i}$};
     \node[tdplot_screen_coords] at (-4.5/10,-.15,0){\begin{minipage}{2.cm}\centering\small \color{red}Incident plane wave\end{minipage}};
\end{tikzpicture}%
