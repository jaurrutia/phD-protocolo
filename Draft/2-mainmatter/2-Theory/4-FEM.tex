% !TeX root = ../tesis.tex

The light scattering problem in its weak formulation, and assuming harmonic time dependency [Eqs. \eqref{eq:Scatt-Weak-All}], can be solved by means of the so-called FEM in the frequency domain, given by Eqs. \eqref{eq:Vec-FEM}. There are several software that allow the user to introduce the desired geometry, physical properties of the system, and boundary conditions in order to calculate the scattered electric field through the FEM. Examples of commercial FEM software with the capability to solve Eqs. \eqref{eq:Scatt-Weak-All} are Altair HyperWorks and CTS StudioSuite as well as open-source alternatives, such as Elmer, OpenFOAM and SCUFF-EM. Nevertheless in this work, the commercial software COMSOL Multiphysics\texttrademark{} Ver. 5.4 (COMSOL) was employed; for a more technical description of the performed COMSOL simulation see Appendix \ref{app:COMSOL}.

The FEM implemented within COMSOL is based on the Galerkin Method [Eq. \eqref{eq:GalerkinMat}] with a variety of finite element families, including, but not limited to, the Lagrange [Ec. \eqref{eq:Lag}], the Hermite [Ec. \eqref{eq:Her}] and the lowest order Nédélec\footnote{Nédélec Elements of higher order are implemented in COMSOL for particular shapes only \cite{comsol_doc}.} families [Ec. \eqref{eq:Nedelec}] \cite{comsol_doc}; such finite element families are built within COMSOL for different shapes in 2D and 3D geometries: triangles, rectangles, pyramids,  prisms, tetrahedrons, and more \cite{comsol_doc}. The core of COMSOL allows the user to set the desired geometry of the PDE problem to be solved, as well as the discretization method and the matrix inversion algorithms to solve Eq. \eqref{eq:GalerkinMat} \cite{comsol_doc}. Additionally, the COMSOL's package Wave Optics implements the Maxwell's eigenvalue problem ---considering harmonic time dependency as in Eqs. \eqref{eq:Scatt-Weak-All} \cite{comsol_doc}--- alongside the physical characteristics of the system: the optical properties, encoded into the electric permittivity and magnetic permeability, of different materials and the several boundary conditions such like the generalized Sommerfeld's radiation condition [Eq. \eqref{eq:SommVec}] or the PML [Eq. \eqref{eq:PMLgen}] \cite{comsol_wave}. The Wave Optics module returns the total electric field and the user can separate it into two contributions: the incident  $\vb{E}^\text{i}$ and the induced  $\vb{E}^\text{ind}$ electric fields; the later corresponds to the scattered (internal) electric field  $\vb{E}^\text{sca}$ ($\vb{E}^\text{int}$) outside (inside) any scatterer \cite{comsol_wave}.

%----------------------------------------------------------------------
\begin{figure}[b!]
	\centering
     \small
     \def\svgwidth{.8\textwidth}
     \hspace*{-.2\textwidth}
       \begin{subfigure}{.2\textwidth}\caption{ }\label{fig:setup:a}\end{subfigure}%
     \hspace*{-6em}%
       \begin{subfigure}{.78\textwidth}\caption{ }\label{fig:setup:b}\end{subfigure}
     \vspace*{-2.5em}\\
    \includeinkscape{Geometries/SistemaBox}
 \vspace*{0em}
\caption[Boxed Particle Setup in COMSOL]{\textbf{a)} Three dimensional view and \textbf{b)} cross section of the geometry employed to solve the light scattering due to a  spherical NP (blue) embedded in a non-absorbing matrix (gray) illuminated by a plane wave in COMSOL; the system is totally covered by a PML (blue) with rectangular geometry. The upper layer of the PML in \textbf{a)} is hidden to allow a better view of the setup.}
\label{fig:setup}
\end{figure}
%----------------------------------------------------------------------
%
To minimize errors that may arise in simulations performed with COMSOL, the analytical solution given by the Mie Theory for the light scattering due to a spherical particle ---introduced in Section \ref{s:Mie}--- was contrasted against an approximated solution returned by COMSOL. Since COMSOL's Documentation \cite{comsol_doc} recommends to let the software to choose the kind of finite element\footnote{The build geometries within COMSOL may require a transformation from the reference finite elements to the real finite elements but COMSOL's internal tools guarantee a non-vanishing Jacobian for such transformations since they are not highly deformed. See \textit{Mesh Element Quality and size} in \cite{comsol_doc}.} to be used, the only numerical characteristics analyzed were the size of the domain $\Omega$, its discretization into finite elements (mesh size)\footnote{COMSOL allows the user to set a minimum and a maximum size of the employed finite elements and these parameters can be chosen as global or local to specific regions \cite{comsol_doc}.}, the discretization of the spherical scatterer, and the thickness of the PML used to simulate an open boundary. The geometry employed for the FEM approximated solution, built with COMSOL's internal tools, is shown in Fig. \ref{fig:setup} where a single spherical NP (blue) is embedded in  the middle of the box-shaped\footnote{The box-shaped geometry was chosen so that in future work, the reflectance and transmittance of the system can be calculated by the internal functions of COMSOL, which requires a planar interface to act as a sensor.} matrix (gray) and this last is covered by a PML (blue) which allows the system to be studied as an infinite non-absorbing medium; the generalized Sommerfeld's radiation condition [Eq. \eqref{eq:SommVec}] was set in addition to the PML since it enhances the performance of the COMSOL simulation. It is worth noting that COMSOL's internal tools proposes a meshing size by default \cite{comsol_doc} and that a thickness for the PML of a fourth of the wavelength $\lambda$ of $\vb{E}^\text{i}$ is recommended \cite{comsol_wave}.

\begin{figure}[b!]
 \def\svgwidth{.9\textwidth}
 \small
 \centering
    \hspace*{-.72\textwidth}
     \begin{subfigure}{\textwidth}\caption{}\label{fig:Eff:First:a}\end{subfigure}\\[11em]
    \hspace*{-.72\textwidth}
     \begin{subfigure}{\textwidth}\caption{}\label{fig:Eff:First:b}\end{subfigure}\\[-14.5em]
     \includeinkscape{FEM/0-NoConv}
\caption[Scattering, Absorption and Extinction Efficiencies of a 12.5 nm AuNP$@$Air: Analytical and FEM solutions with no optimization]{\textbf{a)} Scattering $Q_\text{sca}$ (black), absorption $Q_\text{abs}$ (orange), and extinction $Q_\text{ext}$ (blue) efficiencies of a 12.5 nm AuNP embedded in air calculated by means of the Mie Theory (continuous) and the FEM (markers), and \textbf{b)} their absolute error, as a function of the wavelength $\lambda$ of the incident plane wave.}
\label{fig:Eff:First}
\end{figure}
%

To contrast the solutions obtained analytically and through the FEM, the observable optical quantities in the far-field regime, that is, the scattering $Q_\text{sca}$, absorption $Q_\text{abs}$,  and extinction $Q_\text{ext}$ efficiencies were calculated with both approaches and contrasted. Since the FEM returns the value of the electric field in the whole domain $\Omega$, the Eqs. \eqref{eq:Csca}, \eqref{eq:Cabs} and \eqref{eq:Cext} were employed to calculate $Q_\text{sca}$, $Q_\text{abs}$ and $Q_\text{ext}$, respectively, while they can be calculated with the Mie Theory through Eqs. \eqref{eq:CextSphere} and \eqref{eq:CscaSphere} and the Optical Theorem [Eq. \eqref{eq:Cext}]. In Fig. \ref{fig:Eff:First:a} the efficiencies  $Q_\text{sca}$ (black),  $Q_\text{abs}$ (orange) and $Q_\text{ext}$ (blue) are shown as a function of the incident wavelength $\lambda$, in the visible light regime,  of the incoming electric plane wave $\vb{E}^\text{i}$ that illuminates a AuNP of radius $a = 12.5$ nm (employing the size-corrected dielectric function ---see Appendix \ref{app:SizeCorrection}--- for its optical response) embedded in air (with refractive index $n_\text{m} = 1$); the continuous lines correspond to $Q^\text{Mie}$, the analytical solution calculated by the Mie Theory, and the markers to $Q^\text{FEM}$ the approximated solution returned by COMSOL with the default values for the meshing size ---global parameters applied to the whole domain $\Omega$ and the PML---, the recommended size for the matrix ---the minimum distance between the AuNP and the PML--- and the recommended PML thickness; the values of these parameters are shown in Fig. \ref{fig:Eff:First:a} as inset text. The absolute error (Abs. Error) between the analytical and the FEM  approximated solution, given by $\abs{Q^\text{Mie}-Q^\text{FEM}}$, is shown in Fig. \ref{fig:Eff:First:b}.

The results in Fig. \ref{fig:Eff:First:a} show that the scattering efficiency of a 12.5 nm AuNP, which was multiplied by a factor of 100, is two orders of magnitude smaller than the absorption efficiency, thus making absorption the most important contribution to the light extinction, as discussed in Section \ref{ss:AuMie}. This behavior is reproduced by the FEM simulation as the absolute error of the efficiencies [Fig. \ref{fig:Eff:First:b}] shows a discrepancy between the analytical and the approximated solution in the second digit after the decimal point for $Q_\text{abs}$ and $Q_\text{ext}$, and in the third digit after the decimal point for $Q_\text{sca}$. It is worth noting that the absolute error for the absorption efficiency is maximum at a wavelength $\lambda\sim 509$ nm, which corresponds to the LSPR wavelength, while the absolute error for the scattering efficiency grows linearly with the incident wavelength for $\lambda < 521$ nm, the wavelength of maximum scattering. On the one hand, the absorption efficiency is calculated numerically by integrating $\vb{E}^\text{int}$ in the volume of the AuNP [Eq. \eqref{eq:Cabs}], thus the associated error arises due to the meshing inside the AuNP, which does not resolve $\vb{E}^\text{int}$ good enough for the value of $\lambda$ that maximizes it. On the other hand, the scattering efficiency is obtained by integrating $\vb{E}^\text{sca}$ on a closed surface outside the AuNP [Eq. \eqref{eq:Csca}] ---for the FEM simulations the boundary of the AuNP was chosen as the integration surface---, therefore the associated error is related either to the meshing or to unappropriated implementation of the boundary conditions to simulate an infinite matrix.

%
\begin{figure}[b!]
	\centering
	\def\svgwidth{.9\textwidth}
	\small
	\vspace*{1.em}
	\hspace*{-.75\textwidth}
     \begin{subfigure}{\textwidth}\caption{}\label{fig:Eff:Matrix:a}\end{subfigure}\\[6em]
    \hspace*{-.75\textwidth}
     \begin{subfigure}{\textwidth}\caption{}\label{fig:Eff:Matrix:b}\end{subfigure}\\[-10.75em]
\includeinkscape{FEM/1-Mat-Size-Rad-2}
%\vspace*{-1em}
\caption[Scattering, Absorption and Extinction Efficiencies absolute error: Matrix size analysis ]{Absolute error between the Mie Theory and the FEM approximated solution of the scattering (black), absorption (orange) and extinction (blue) efficiencies of a 12.5 AuNP embedded \textbf{a)} in an air matrix ($n_\text{m} = 1$) and \textbf{b)} in a glass matrix ($n_\text{m} = 1.5$) as a function of the matrix size. The FEM numerical simulation parameters were the default mesh size ---maximum (minimum) element size of 110 nm (3.3 nm) globally--- and the recommended $\lambda/4$ PML thickness. }
\label{fig:Eff:Matrix}
\end{figure}
%
To analyze the effect of the size of the matrix on the numerical convergence of the FEM approximated solution, its absolute error against the analytical Mie Theory solution is shown in Fig. \ref{fig:Eff:Matrix} for $Q_\text{sca}$, $Q_\text{abs}$ and $Q_\text{ext}$, for two different systems: a 12.5 nm AuNP embedded in air [Fig. \ref{fig:Eff:Matrix:a}] and in glass [Fig. \ref{fig:Eff:Matrix:b}]. The efficiencies are evaluated at the LSPR wavelength (continuous lines) ---507 nm for air and 533 nm for glass--- and at the wavelength of maximum scattering (dash lines) ---521 nm for air and 543 nm for glass--- where the error was maximum and where it started to grow, as seen in Fig. \ref{fig:Eff:First:b}. In Fig. \ref{fig:Eff:Matrix} the efficiencies are plotted as a function of the minimum distance between the AuNP's surface and the PML, that is, as a function of the size of the matrix (in the upper frame the distance is measured in units of the AuNP's radius $a=12.5$ nm) and the chosen values for the element size and the PML thickness for the FEM simulation were the same as those in Fig. \ref{fig:Eff:First}.

%
\begin{figure}[b!]
	\centering
	\def\svgwidth{.9\textwidth}
	\small
	\hspace*{-.77\textwidth}
	 \begin{subfigure}{\textwidth}\caption{}\label{fig:Eff:PML:a}\end{subfigure}\\[7.35em]
	\hspace*{-.77\textwidth}
	      \begin{subfigure}{\textwidth}\caption{}\label{fig:Eff:PML:b}\end{subfigure}\\[-11.75em]
\includeinkscape{FEM/2-PML-Size}
\caption[Scattering, Absorption and Extinction Efficiencies absolute error: PML thickness analysis ]{Absolute error between the Mie Theory and the FEM approximated solution of the scattering (black), absorption (orange) and extinction (blue) efficiencies of a 12.5 AuNP embedded \textbf{a)} in an air matrix ($n_\text{m} = 1$) and \textbf{b)} in a glass matrix ($n_\text{m} = 1.5$) as a  function of the PML thickness. The FEM numerical simulation parameters were the default mesh size ---maximum (minimum) element size of 110 nm (3.3 nm) globally--- and a matrix width of $(15+1) a n_\text{m}$.}
\label{fig:Eff:PML}
\end{figure}

The absolute error on the absorption and extinction efficiencies remain in the same order of magnitude independently of the matrix size for any matrix material and chosen wavelength, as it can be seen from the orange and blue lines in Fig. \ref{fig:Eff:Matrix}. However, the scattering efficiency does decrease its absolute error, linearly, as the matrix size grows  up to a critical value, $\sim 15 a/2$ and $\sim 10 a/2$ for an air and a glass matrix, respectively, after which the relative error converges. The gray shaded region in Figs. \ref{fig:Eff:Matrix:a}  and \ref{fig:Eff:Matrix:b} corresponds to the converged regime for each matrix material. Since the ratio between the critical value of the matrix size between the air  and the glass matrix system is approximately $1.5$, the refractive index of glass, a matrix of width $\sim (15+1) a n_\text{m}$ guarantees a minimum in the scattered electric field $\vb{E}^\text{sca}$ due to the size of the matrix. This result is only valid for plasmonic NPs, such as the AuNPs, small compared to the incident wavelength, since the scattering is the least important contribution to extinction, as seen in Fig. \ref{fig:Eff:First:a}. The convergence of the absolute error on the scattering efficiencies is due to the Sommerfeld's radiation condition [Eq. \eqref{eq:SommVec}], which is an enough strong criteria to simulate an open boundary if the matrix is sufficiently large; this is supported by the absolute errors shown in Fig. \ref{fig:Eff:PML} where they are plotted as a function of the PML thickness considering a default meshing size ---see values in Fig. \ref{fig:Eff:First:a}--- and a matrix width of  $(15+1) a n_\text{m}$ for both an air [Fig. \ref{fig:Eff:PML:a}] and a glass [Fig. \ref{fig:Eff:PML:b}] matrix system, where none of the shown absolute errors change considerably.

From the analysis of Figs. \ref{fig:Eff:Matrix} and \ref{fig:Eff:PML}, the absolute error on the scattering efficiency obtained with the default COMSOL parameters [Fig. \ref{fig:Eff:First}] was decreased for a plasmonic NP small compared to the incident wavelength, such as a 12.5 AuNP illuminated in the visible range. To diminish the absolute error on the absorption efficiency, and thus on the extinction, a variation of the mesh size inside the AuNP was performed, while setting the default mesh size in the matrix but considering a recommended PML thickness of $\lambda/4$ and a matrix width of $(15+1)an_\text{m}$. In Fig. \ref{fig:Eff:rad} the absolute error of the scattering, absorption and extinction efficiencies are shown as a function  of the maximum mesh size inside a 12.5 AuNP embedded in air [Fig. \ref{fig:Eff:rad:a}] and in glass [Fig. \ref{fig:Eff:rad:b}], at the wavelength of the LSPR and at the maximum scattering wavelength for each system. The variation of the minimum mesh size is shown in nanometers for the lower frame side and  in fractions of the AuNP's radius for the upper frame side.

\begin{figure}[h!]
	\centering
	\def\svgwidth{.9\textwidth}
	\small
	\vspace*{1em}
	\hspace*{-.47\textwidth}
	 \begin{subfigure}{\textwidth}\caption{12.5 AuNP$@$Air}\label{fig:Eff:rad:a}\end{subfigure}\\[5.5em]
	\hspace*{-.45\textwidth}
	 \begin{subfigure}{\textwidth}\caption{12.5 AuNP$@$Glass}\label{fig:Eff:rad:b}\end{subfigure}\\[-10.1em]
\includeinkscape{FEM/3-Rad-Mesh}
\caption[Scattering, Absorption and Extinction Efficiencies absolute error: meshing size inside AuNP analysis ]{Absolute error between the Mie Theory and the FEM approximated solution of the scattering (black), absorption (orange) and extinction (blue) efficiencies of a 12.5 AuNP embedded \textbf{a)} in an air matrix ($n_\text{m} = 1$) and \textbf{b)} in a glass matrix ($n_\text{m} = 1.5$) as a function of the mesh size inside the AuNP. The FEM numerical simulation parameters were the default mesh size  in the matrix---maximum (minimum) element size of 110 nm (3.3 nm)---, a matrix width of $(15+1) a n_\text{m}$ and a PML thickness of $\lambda/4$.}
\label{fig:Eff:rad}
\end{figure}

As it can be seen in Fig. \ref{fig:Eff:rad}, the absolute error of the efficiencies decreases exponentially as the mesh size inside the AuNP becomes finer for all the considered cases: air and glass matrices at two different wavelengths each. Such behavior can be explained since the absorption (scattering) efficiency is obtained by integrating the internal (scattered) electric field at the AuNP's volume (surface). Thus, a finer mesh at the scatterer allows the induced electric field to be not only better resolved by the FEM, but also to increase the accuracy of the performed numerical integration. Additionally,  the results in Fig. \ref{fig:Eff:rad} show that the decrement of the absolute error on $Q_\text{abs}$ has a higher rate than the decrement on the absolute error of $Q_\text{sca}$ due to the amount of points in the mesh inside the volume and on the surface of the AuNP. Lastly, the refinement of the mesh size inside the AuNP, alongside with the chosen matrix width and PML thickness, guarantees a difference between the analytical Mie Theory solution and the approximated FEM solution up to the fourth digit after the decimal point both on the scattering and absorption efficiencies. If a refinement in the mesh side inside the matrix were to be done, the absolute error on the efficiencies may diminish even further, but such refinement may not be appreciable for the goals of this thesis. Therefore, the recommended mesh side in the matrix ---maximum element size of $\lambda/(6n_\text{m})$ for first order finite elements \cite{comsol_doc}--- is used as it is smaller than the default mesh size in this region.

Taking into account the results shown of Figs. \ref{fig:Eff:Matrix}--\ref{fig:Eff:rad} and the recommended maximum mesh size in the matrix, a converged result of the scattering, absorption and extinction efficiencies by means of the FEM is obtained. In Fig. \ref{fig:Eff:Conv} the optical efficiencies of a $12.5$ nm AuNP embedded in an air and a glass matrix are shown as a function of the incident wavelength $\lambda$, as well as the relative error between the analytical (lines) and FEM (markers) results: the continuous lines and solid markers correspond to the air matrix case while  the dashed lines and open markers to the glass matrix case.

\begin{figure}[h!]
 \def\svgwidth{.95\textwidth}
 \small
 \centering
    \hspace*{-.78\textwidth}
     \begin{subfigure}{\textwidth}\caption{}\label{fig:Eff:Conv:a}\end{subfigure}\\[10.85em]
    \hspace*{-.78\textwidth}
     \begin{subfigure}{\textwidth}\caption{}\label{fig:Eff:Conv:b}\end{subfigure}\\[-14.3em]
     \includeinkscape{FEM/4-Conv}
\caption[Scattering, Absorption and Extinction Efficiencies of a 12.5 nm AuNP$@$Air: Analytical and FEM solutions with optimized parameters for a FEM simulation]{\textbf{a)} Scattering $Q_\text{sca}$ (black), absorption $Q_\text{abs}$ (orange) and extinction $Q_\text{ext}$ (blue) efficiencies of a 12.5 nm AuNP embedded in air calculated by means of the Mie Theory (continuous lines) and the FEM (solid markers) and embedded in glass (dashed lines and open markers), and \textbf{b)} their absolute error, as a function of the wavelength $\lambda$ of the incident plane wave. The FEM numerical simulation parameters were the recommended mesh side in the matrix ---maximum element size of $\lambda/(6n_\text{m})$--- and a maximum mesh side inside the AuNP of $a/5$, a matrix width of $(15+1) a n_\text{m}$ and a PML thickness of $\lambda/4$.}
\label{fig:Eff:Conv}
\end{figure}

By comparing the absolute error employing COMSOL's default values for the FEM simulation [Fig. \ref{fig:Eff:First:b}] with optimized the parameters [Fig. \ref{fig:Eff:Conv:b}] considering an air matrix, it can be seen that the later is an order of magnitud smaller for the absorption and extinction efficiencies in the visible spectrum, while the absolute error of the scattering efficiency between these choices of parameters remains the same nevertheless, the FEM simulation with the optimized parameters required less computational resources than with the default COMSOL parameters due to the smaller matrix size. Additionally, the absolute error of the optical efficiencies considering a AuNP embedded in a glass matrix are smaller for all efficiencies in the visible spectrum than those of a AuNP embedded in an air matrix. This behavior is explained by the nominal values of the optical efficiencies [Fig. \ref{fig:Eff:Conv:a}] since the greater they are, the greater the induced electric field, that is, the induced electric fields are better resolved for more optically dense media and thus their absolute error diminishes.
