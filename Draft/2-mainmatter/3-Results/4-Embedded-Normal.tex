% !TeX root = ../tesis.tex

To study the optical properties of a partially embedded spherical AuNP into a substrate forming a planar interface with a matrix, the incrustation parameter is introduced, which is defined as the ratio $h/a$, where $h$ is the vertical position of the center of the AuNP relative to the interface and $a$ its radius, as it can be seen in Fig. \ref{fig:Inc:Eff}. The choice of the incrustation parameter is based on the volume $V_\text{sph}^\text{sub}$ (surface $S_\text{sph}^\text{sub}$) fraction of the
AuNP partially embedded relative to the total volume $V_\text{sph}$ (surface $S_\text{sph}$) of the AuNP   obbeying the relation $V_\text{sph}^\text{sub}/V_\text{sph} = S_\text{sph}^\text{sub}/S_\text{sph} = (1-h/a)/2$. Such a definition of the incrustation parameter $h/a$ characterizes the totally supported and totally embedded AuNP  ---see Section \ref{s:Totally:Normal}--- by the values $1$ and $-1$, respectively, while in general  if $h/a>0$ it is understood that most of the AuNP is located in the matrix and if $h/a<0$ most of the AuNP is buried in the substrate.

In Fig. \ref{fig:Inc:Eff} the absorption $Q_\text{abs}$ [Fig.\ref{sfig:IncNormal:1}] and scattering $Q_\text{sca}$ [Fig. \ref{sfig:IncNormal:2}] efficiencies of a 12.5 nm AuNP partially embedded in a glass substrate ($n_\text{s} = 1.5$) forming a planar interface with an air matrix ($n_\text{m} = 1$), are shown as a function of the wavelength $\lambda$ of the incident electric field $\vb{E}^\text{i}$ illuminating the system at normal incidence in an internal configuration ---as shown in Fig. \ref{fig:Inc:Eff}--- for several values of the incrustation parameter. The magenta markers correspond to the values of $Q_\text{abs}$ and $Q_\text{sca}$ evaluated at the wavelength of resonance and the black, light brown, and purple curves correspond to the limiting values of $h/a$ equal to 1 (totally supported NP), -1 (totally embedded NP) and 0 (half of the NP in the substrate and half in the matrix). To compare the obtained results, the Mie-limiting cases (AuNP embedded in air and in glass)  are signalized in Fig. \ref{fig:Inc:Eff} by the boundaries of the green shaded region  and the cyan markers correspond to the resonances of their efficiencies; the gray dashed lines are guides to the eye.

Both the absorption and the scattering efficiencies for partially embedded 12.5 AuNP present an increase  for all wavelengths as the AuNP is buried into the substrate, that is, as the incrustation parameter changes from 1 to -1. Additionally, the  absorption wavelength of resonance is redshifted  increasingly from $510$ nm to $532.5$ nm, as it can be seen from the gray dashed line joining the magenta markers for  $Q_\text{abs}$  with $h/a = 1$ (black curve) and $h/a = -1$ (light brown curve) in Fig. \ref{sfig:IncNormal:1}. On the other hand, the wavelength of resonance for $Q_\text{sca}$ presents an increasing redshift for its wavelength of resonance when $h/a$ increases, \textcolor{black}{except when the scattering efficiency of the partially embedded AuNP with $h/a = 0$ (purple curve) and with $h/a = -0.25$ (green curve) in Fig. \ref{sfig:IncNormal:2} are compared since for these two cases there is a blueshift of the resonance rather than a redshift; all of which is true up to the wavelength discretization of $\Delta\lambda = 2.5$ nm employed in the FEM simulations in the neighborhood of the resonances}. Lastly, the absorption and scattering efficiencies for a partially embedded AuNP present only one resonance within the visible range that is spectrally located in between the resonances for the Mie-limiting cases (AuNP in an homogeneous matrix of air and of glass), therefore its optical properties can be mostly described by dipolar contribution, even if the homogeneity of its surroundings is broken as well as the symmetry of the system. To emphasize the last observation, the radiation pattern and the spatial distribution of the induced electric field are to be analyzed.

\begin{figure}[h!]
    \def\svgwidth{\textwidth}
    \small
    \includeinkscape{3-IncNorm/1-Efficiencies/1-Normal-Eff} \\[-15em]
    \hspace*{-.375\textwidth}%
        \begin{subfigure}{.765\textwidth}\caption{ }\label{sfig:IncNormal:1}\end{subfigure}%
        \begin{subfigure}{.25\textwidth}\caption{ }\label{sfig:IncNormal:2}\end{subfigure}\\[11.5em]
    \caption[Absorption and Scattering Efficiencies of a partially embedded 12.5 nm AuNP into a substrate Illuminated in an internal configuration at normal incidence]{\textbf{a)} Absorption and \textbf{b)} scattering efficiencies of a $12.5$ nm AuNP partially embedded in a glass substrate ($n_\text{s} = 1.5$) with an air matrix ($n_\text{m} = 1$) as a function of the wavelength $\lambda$ of the incident electromagnetic plane wave with a wave vector $\vb{k}^\text{i}$ perpendicular to the glass-air interface. The partial embedding of the AuNP is determined by the ratio $h/a$ with $a$ the AuNP's radius and $h$ the distance between the interface and the center of the AuNP. The green shaded region shows the two Mie-limiting cases of a AuNP embedded either in air or in glass; the magenta (partially embedded AuNP) and cyan (Mie-limiting) markers correspond to the efficiencies evaluated at the wavelength of resonance for each case; the gray dashed line is a guide to the eye.}
    \label{fig:Inc:Eff}
\end{figure}

The radiation patterns of a partially embedded AuNP in a glass substrate and an air matrix, when it is illuminated by a plane wave  at normal incidence, are shown in Fig. \ref{fig:Far:IncNorm} for several values of the incrustation parameter of $h/a$ evaluated at the resonance wavelength of the LSPR [magenta markers in Fig. \ref{sfig:IncNormal:1}]. The  radiation patterns are evaluated at the scattering planes considering an incident electric field parallel  $\vb{E}^\text{i}_\parallel$ [Fig. \ref{sfig:Far:IncNorm:a}] and  perpendicular  $\vb{E}^\text{i}_\perp$ [Fig. \ref{sfig:Far:IncNorm:b}] to it. Additionally, the color code employed for each value of $h/a$ is the same as in Fig. \ref{fig:Inc:Eff}.

\begin{figure}[h!]
    \centering
    \def\svgwidth{.8\textwidth}
    \includeinkscape[pretex = \footnotesize]{3-IncNorm/4-5-FarXY/4-5-Far-XY}\\
        \vspace*{-16.5em}%
        \hspace*{-.2\textwidth}%
    \begin{subfigure}{.4\textwidth}\caption{\footnotesize$\dfrac{\norm{\vb{E}^\text{sca}_\text{far}}}{\norm{\vb{E}^\text{i}}} \; [10^{-9}]$  }\label{sfig:Far:IncNorm:a}\end{subfigure}%
    \begin{subfigure}{.4\textwidth}\caption{\footnotesize$\dfrac{\norm{\vb{E}^\text{sca}_\text{far}}}{\norm{\vb{E}^\text{i}}} \; [10^{-9}]$  }\label{sfig:Far:IncNorm:b}\end{subfigure}\\[14em]
    %
    \caption[  Radiation pattern of a AuNP supported on a substrate illuminated at oblique incidence ]{Radiation patterns of a AuNP (light yellow) of radius $a = 12.5$ nm partially embedded in a glass substrate ($n_\text{s} = 1.5$, light blue) in an air matrix ($n_\text{m} = 1$) illuminated by an incident electric plane wave $\vb{E}^\text{i}$, with a wavelength $\lambda$ traveling in the $\vb{k}^\text{i}$ direction, normal to the glass-air interface. The radiation patterns consider the incident electric field \textbf{a)} $\vb{E}^\text{i}_\parallel$ parallel to the scattering plane  and \textbf{b)} $\vb{E}^\text{i}_\perp$ perpendicular to it. The partial embedding of the AuNP is determined by the ratio $h/a$, with $h$ the distance between the interface and the center of the AuNP and $a$ its radius; each  radiation pattern is evaluated at the wavelength of the LSPR shown in Fig. \ref{sfig:IncNormal:1}.}
    \label{fig:Far:IncNorm}
\end{figure}

Similar to the radiation patterns at normal incidence for totally embedded and supported 12.5 nm AuNPs (Section \ref{s:Totally:Normal}), the radiation patterns of partially embedded AuNPs present the characteristic two and one-lobe shapes of dipolar radiation, with an asymmetrical behavior in the $\vb{k}^\text{i}$ direction due to the presence of the substrate, when  the incident electric field has a component only parallel and only perpendicular to it, as it can be seen in Figs. \ref{sfig:Far:IncNorm:a} and \ref{sfig:Far:IncNorm:b}, respectively. Another common characteristic, already discussed in the past Sections, is that the average magnitude of the scattered electric field is modulated by the efficiencies shown in Fig. \ref{fig:Inc:Eff}. For example, the AuNP with an incrustation parameter $h/a = -0.75$ (light purple curves) has a maximum amplitude in the far-field regime of $\sim  2.3  $ while this value decreases to $\sim 1.5$ for $h/a = 0$ (purple curves), whose ratio resembles that of the absorption efficiencies of $\sim 2.7/1.6$ ---see magenta markers in Fig. \ref{sfig:IncNormal:1}--- for the same incrustation parameters. The overall behavior of the absorption and scattering efficiencies and of the radiation patterns from Figs. \ref{fig:Inc:Eff} and \ref{fig:Far:IncNorm} suggests that the small particle approximation, where the 12.5 nm AuNP can be described as an electric point dipole, is still valid even if the AuNP is partially embedded so far it is illuminated normally to the glass-air interface, which can be confirmed by analyzing the induced electric field in the near-field regime.

\begin{figure}[h!]\centering
   \def\svgwidth{.75\textwidth}
   \footnotesize
   \captionsetup[subfigure]{labelfont ={normal,bf,color = white}}
   \includeinkscape{3-IncNorm/2-Near/2-Near}\\[-47.5em]
   \hspace*{-.25\textwidth}
       \begin{subfigure}{.25\textwidth}\caption{ } \label{sfig:Near:IncNorm:50:par}\end{subfigure}%
       \begin{subfigure}{.34\textwidth}\caption{ }\label{sfig:Near:IncNorm:50:perp}\end{subfigure}\\[13em]
    \hspace*{-.25\textwidth}
        \begin{subfigure}{.25\textwidth}\caption{ }\label{sfig:Near:IncNorm:00:par}\end{subfigure}%
        \begin{subfigure}{.34\textwidth}\caption{ }\label{sfig:Near:IncNorm:00:perp}\end{subfigure}\\[13em]
   \hspace*{-.25\textwidth}
       \begin{subfigure}{.25\textwidth}\caption{ } \label{sfig:Near:IncNorm:-5:par}\end{subfigure}%
       \begin{subfigure}{.34\textwidth}\caption{ }\label{sfig:Near:IncNorm:-5:perp}\end{subfigure}\\[15em]
   \caption[Induced Electric Field of a 12.5 nm Au Spherical NP embbeded into (supported on) a substrate illuminated at a normal incidence]{
       Magnitude of the electric field $\vb{E}^\text{ind}$ induced by a partially embedded 12.5 nm AuNP (dashed black lines) illuminated by an incident electromagnetic plane wave $\vb{E}^\text{i}$ traveling in the $\vb{k}^\text{i}$ direction perpendicular to the interface ---white dashed lines--- between an air matrix ($n_\text{m} = 1$) and a glass substrate ($n_\text{s} = 1.5$). The incident electromagnetic plane wave is evaluated at the resonance wavelength of the absorption efficiency ---see Fig. \ref{sfig:IncNormal:1}--- for an incrustation parameter $h/a$ of \textbf{a,b)} $0.50$, \textbf{c,d)} $0$ and \textbf{e,f)} $-0.50$, and considering an incident electric field \textbf{a,c,e)} $\vb{E}^\text{i}_\parallel$ parallel to the scattering plane and \textbf{b,d,f)} $\vb{E}^\text{i}_\perp$ perpendicular to it.
   }
   \label{fig:Near:IncNorm}
 \end{figure}

In Fig. \ref{fig:Near:IncNorm} it is shown the spatial distribution of the magnitude of the induced electric field $\vb{E}^\text{ind}$ due to the interaction between a plane wave traveling normally to the interface between air and glass (horizontal white dashed lines), where a 12.5 nm AuNP (black dashed lines) is partially embedded with a incrustation parameter $h/a = 0.50$ [Figs. \ref{sfig:Near:IncNorm:50:par} and \ref{sfig:Near:IncNorm:50:perp}], $h/a = 0$ [Figs. \ref{sfig:Near:IncNorm:00:par} and \ref{sfig:Near:IncNorm:00:perp}], and $h/a = - 0.50$ [Figs. \ref{sfig:Near:IncNorm:-5:par} and \ref{sfig:Near:IncNorm:-5:perp}]. The magnitude of $\vb{E}^\text{ind}$ is shown for the scattering plane parallel to the incident electric field [Figs. \ref{sfig:Near:IncNorm:50:par}, \ref{sfig:Near:IncNorm:00:par}, and \ref{sfig:Near:IncNorm:-5:par}] and perpendicular to it [Figs. \ref{sfig:Near:IncNorm:50:perp}, \ref{sfig:Near:IncNorm:00:perp}, and \ref{sfig:Near:IncNorm:-5:perp}] considering that the incident electromagnetic plane wave has a wavelength $\lambda$ equal to the wavelength of resonance for each case.

The spatial distribution of the induced electric field shows two hotspots for each value of the incrustation parameter considered ---with a sixfold enhancement-- when the incident electric field is parallel to the scattering plane [see reddish regions in Figs.  \ref{sfig:Near:IncNorm:50:par}, \ref{sfig:Near:IncNorm:00:par} and \ref{sfig:Near:IncNorm:-5:par}], which are not aligned to the equator of the AuNP but located on the surface of the AuNP embedded in the substrate. Such behavior was observed in Section \ref{s:Totally:Normal} where the enhancement of the electric field was larger for the totally embedded AuNP. When the AuNP is partially  embedded, the larger enhancement in the substrate is not only observed in the hostspots but also in the norm of the scattered electric field reaching a farther region in the glass substrate than in the air matrix, as it can be seen by comparing the bluish regions in Figs. \ref{sfig:Near:IncNorm:50:perp}, \ref{sfig:Near:IncNorm:00:perp} and \ref{sfig:Near:IncNorm:-5:perp}, where the scattering plane is perpendicular to the incident electric field, above and below the white dashed lines. While this behavior of $\norm{ \vb{E}^\text{ind}}$ is common to all values of the incrustation parameter $h/a$, the effect of $h/a$ can be identified in the localization of the hotspots and in the internal electric field inside the AuNP: the hotspots are less localized as $h/a$ changes from $1$ to $-1$ and the internal electric field is greater in the regions where the AuNP is in the substrate than in the matrix. The lesser localization of the hostspots the larger the incrustation parameter, is understood by taking into account the polarization of the materials (glass and air), while the magnitude of the gradient for the internal electric field is due to the boundary conditions on the AuNP's surface.

From the analysis of the absorption $Q_\text{abs}$ and scattering $Q_\text{sca}$ efficiencies (Fig. \ref{fig:Inc:Eff}), and of the induced electric field in the far (Fig. \ref{fig:Far:IncNorm}) and near-field  (Fig. \ref{fig:Near:IncNorm})  regimes, it was identified a similar behavior of the optical properties of a 12.5 nm AuNP when it is totally supported on (embedded in)  a substrate  than when the AuNP is partially embedded in bewteen the substrate and the matrix, for a normal incidence of an electromagnetic plane wave. Specifically, it was found that the optical properties of the AuNP can be mostly described by a dipolar contribution with any value  of the incrustation parameter, whose effect is noticeable in a redshift of the LSPR and an enhancement of $Q_\text{abs}$  and $Q_\text{sca}$ as the AuNP is more embedded into the substrate, as well as a stronger scattered field in the far-field regime and larger hotspots in the near-field regime on the surface of the AuNP embedded in the substrate. The later, shows a disadvantage of the partial embedding of the AuNP for its possible application as a biosensor since the largest enhancement occurs at a spatial region inaccessible to any desired sample. In the following Section, the optical properties of partially embedded AuNP are studied at oblique incidence in order to find an optimal configuration where the partially embedded AuNP is suited for interactions in the matrix side.
