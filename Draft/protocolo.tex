%%%%%%%%%%%%%%%%%%%%%%%%%%%%%%%%%%%%%%%%%%%%%%%%%%%%%%%%%%%%%%%%%%%%%%%%%%%%%%%%
%                     Bachelor thesis template
%						 Science Faculty
%								at
%			National Autonomous University of Mexico (UNAM)
%%%%%%%%%%%%%%%%%%%%%%%%%%%%%%%%%%%%%%%%%%%%%%%%%%%%%%%%%%%%%%%%%%%%%%%%%%%%%%%%
% based on Harish Bhanderi's PhD/MPhil template, then Uni Cambridge
% http://www-h.eng.cam.ac.uk/help/tpl/textprocessing/ThesisStyle/
% corrected and extended in 2007 by Jakob Suckale, then MPI-iCBG PhD programme
% and made available through OpenWetWare.org - the free biology wiki
%
%                     Under GNU License v3
%
% Adapted for the Engineering School at UNAM by Jesús Velázquez y Marco Ruiz
% Then, adapter fot he Science Faculty at UNAM by Jonathan Urrutia
%
%
%
% All used packages are found in
%
%					./Latex/Classes/PhDthesisPSnPDF.cls
%
% within this last document, there are some lines to be UnCommented for a printed or a digital version of the output file:
%
%			line 184
%			lines 231-261hfill
%
%	Since this is a template for a thesis made at UNAM (Mexico) the titles may be in spanish but within PhDthesisPSnPDF.cls it can be change
%

\documentclass[11pt]{Latex/Classes/PhDthesisPSnPDF}

\usepackage{blindtext}
\input{Latex/Comands}           % Special commands written by the author


\svgpath{{1-Theory/figs},{5-Apendices/figs},{2-Results/figs}}
\graphicspath{{5-Apendices/figs},{0-5-Introduccion/figs}}
\lstset{basicstyle=\ttfamily\footnotesize,breaklines=true}


\usepackage{Latex/mmacells}
\usepackage{rotating}

\mmaSet{
  morefv={gobble=2},
  linklocaluri=mma/symbol/definition:#1,
  morecellgraphics={yoffset=1.9ex}
}
\allowdisplaybreaks
\DeclareMathAlphabet\mathbfcal{OMS}{cmsy}{b}{n}

%%-------------------------------------------------------------------------------
%                                  Information of the Student
%-------------------------------------------------------------------------------
% --- Default information for Bachelor's degree
%\author{Jonathan Alexis Urrutia Anguiano}
%\title{Optical response of partially embedded nanospheres}
%\programa{Posgrado en Ciencias Físicas} % Licenciatura en Física
%\degree{Maestro en Ciencias}% Maestro en / Doctor en
%\director{Dr. Alejandro Reyes Coronado}% Thesis director
%\facultad{Facultad de Ciencias}
%\lugar{Ciudad de México, México}% Place of the dissertation
%\degreedate{2021}% Year of the dissertation
%\portadatrue   %Uncomment for color cover

% ----------------------------- Datos del jurado de Licenciatura
%\student{Paternal last name\\ Maternal Last name\\ Names\\ Telephone number\\ Universidad Nacional Autónoma de México\\ Facultad de Ciencias\\ Física\\ Student Number}
%\secretario{Dr \\ Secretary (thesis director) \\ Last name \\ Last name}
%\presidente{Dr \\ President \\ Last name \\ Last name}
%\vocal{Dr \\ Vocal \\ Last name \\ Last name}
%\supuno{Dr \\ substitute 1 \\ Last name \\ Last name}
%\supdos{Dr \\ Substitute 2 \\ Last name \\ Last name}
%\pags{pages}


% --- Default information for Grad degree

\posgradotrue
	 \author{Jonathan Alexis Urrutia Anguiano}
		 \title{Titulo que falta definir}
		 \programa{Posgrado en Ciencias Físicas} 	% Programa de posgrado
		 \campoconocimiento{Óptica y Fotónica}
		  \degree{Candidatura  al grado de Doctor en Ciencias}				% Maestro en / Doctor en

	\director{Dr. Alejandro Reyes Coronado}		% Thesis director
	\directordep{Facultad de Ciencias, UNAM}

	\lugar{Ciudad de México}			% Place of the dissertation
	\degreedate{Mayo de 2025} 							% Year of the dissertation  - Arreglar espacio inicial
	\campo{Física}								% Area del posgrado

	\comitetrue									% Comité académico en la portada
											% Falta ver si se ponen extra dentro del documento
		\ctutoruno{Dr. Rubén Ramos García}
		\ctutorunodep{Instituto Nacional de Astrofísica, Óptica y Electrónica}
		\ctutordos{Dr. Wolf Luis Mochán Backal}
		\ctutordosdep{Instituto de Ciencias Físicas, UNAM}

\keywords{Plasmonics, LSPR, Partially embedded, Gold, Finite Element Method, COMSOL}            % For metadata
\subject{PhD protocol, multiple scattering, metasurfaces}                     % Subjects for metadata














%-------------------------------------------------------------------------------
%                                   COVER
%-------------------------------------------------------------------------------
\begin{document}
%
\maketitle
%-------------------------------------------------------------------------------
%                                   FRONT MATTER
%-------------------------------------------------------------------------------
\frontmatter

%\include{0-1-Acknowledgments/1-Acknowledgments}
%\include{0-2-Declaration/1-Declaration}
%\include{0-3-Quote/1-Quote}
\include{0-4-Resumen/1-Resumen}

%-------------------------------------------------------------------------------
%                                INDICES                                    |
%-------------------------------------------------------------------------------
%
\setcounter{secnumdepth}{3} % organizational level that receives a numbers
\setcounter{tocdepth}{3}    % print table of contents for level 3
 
\tableofcontents            % Print main index


%-------------------------------------------------------------------------------
%                                MAIN MATTER
%-------------------------------------------------------------------------------
% the main text starts here with the introduction, 1st chapter,...
\mainmatter

\def\baselinestretch{1}                   % Line spacing

\input{0-5-Introduccion/1-Motivation-Structure}

%\part{Theoretical Framework}
\chapter{Scattering Theory of a Single Spherical Particle}
  \label{ch:OpticalProperties}

	\section{The Optical Theorem: Amplitude Matrix and Cross Sections}
	 \label{s:AmpMatCrossSect}
	 \input{1-Theory/1-OpticalTheorem}


%     In the previous Section, it was concluded that the extinction of light due to the interaction between a particle and a monochromatic plane wave can be determined through the amplitude of the scattered field in the forward direction. This is stated in the Optical Theorem, which is an exact relation, but inaccuracies may arise when either the scattering amplitude matrix or the extinction cross section are approximated\footnote{See for example Section 2.4 from Ref. \cite{tsang_scattering_2000} on the Rayleigh Scattering, and Section 21.7 from Ref. \cite{zangwill_modern_2013} on Thompson scattering.}. A particular case in which the scattering amplitude matrix can be exactly calculated is when the scatterer has spherical symmetry. In order to address this special case, it will be introduced a vectorial basis with spherical symmetry, known as the Vector Spherical Harmonics (VSH). Once the VSH are defined, they will be used to write a monochromatic plane wave in terms of the VSH. By imposing the continuity of the tangential components of the electric and magnetic fields, the scattered field can be also written in terms of the VSH. As a particular example of interest, shown in the last Section, the optical properties of a gold nanoparticle with radius of $12.5$ nm are calculated.
%
%     % ------------------------------- index entries --------
%     \index{Scattering!Thompson}%
%     \index{Scattering!Mie}%
%     \index{Scattering!Rayleigh}%
%     \index{Vector!Spherical Harmonics}%

\chapter{Results and Discussion}
	\label{ch:Results}

    \section{Supported and Totally Embedded Spherical Particles}
     \label{s:Totally}
        \input{2-Results/0-Intro}

        \subsection{Normal Incidence}
        \input{2-Results/1-Totally}
         \label{s:Totally:Normal}
       

\chapter*{Conclusions}
\addcontentsline{toc}{chapter}{\protect\numberline{}Conclusions}
	\label{ch:Conclusions}
	\input{3-Conclusions/1-Conclusions}	
	

\appendix

\chapter{Mie Theory (Conventions)}
  \label{app:MieCode}
  \input{5-Apendices/1-SpecialFunc}


%%-------------------------------------------------------------------------------
%%                               References                                   |
%%-------------------------------------------------------------------------------
%
%\appendix
%\input{5-Apendices/1-Something_extra.tex}

\setlength\bibitemsep{.1\itemsep}
\printbibliography

\newpage
%: ----------------------- list of figures/tables ------------------------
%\listoffigures              % Genera el ínidce de figuras, comentar línea si no se usa
%\listoftables               % Genera índice de tablas, comentar línea si no se usa
\printindex
%-------------------------------------------------------------------------------
%                              Appendix                                   |
%-------------------------------------------------------------------------------



\end{document}
