% !TeX root = ../tesis.tex

% Thesis Abstract -----------------------------------------------------

%\begin{abstractslong}    %uncommenting this line, gives a different abstract heading
\begin{abstracts}        %this creates the heading for the abstract page
\addcontentsline{toc}{chapter}{\protect\numberline{}Abstract}
\vfill
\small
Plasmonic metasurfaces, metallic nanostructures supported on a substrate, have been used as alternatives for biosensing due to their low-cost and easy-to-use features, and due to their light enhancement and confinement capacity. In common biosensing techniques, a liquid flows over the substrate, where the nanostructure is located, so there is a  detachment risk. Therefore, a partial embedding of the nanostructure in the substrate is desirable, which modifies its optical response under ideal conditions. In this thesis, it is studied the optical response of a single spherical gold nanoparticle of radius 12.5 nm, suited for biosensing-aimed-metasurfaces, when the nanosphere is partially embedded in between an air matrix and glass substrate, both which form a flat interface, and illuminated by an electromagnetic plane wave with wavelengths in the optical range, considering two states of polarization as well as different angles of incidence. The optical properties of the partially embedded nanosphere, that is, the scattering, absorption and extinction cross sections and the induced electric field in the near and far-field regimes, are calculated by means of the Finite Element Method and compared with the analytical solutions of two limiting cases: a nanosphere embedded in an infinite matrix of air, and in an infinite matrix of glass. Based on the obtained numerical results, it was determined optimal configurations for  biosensing with a disordered metasurface of partially embedded nanosphere of radius 12.5 nm in the diluted regime.\\[2em]

Las metasuperfices plasmónicas, nanoestructuras metálicas soportadas por un sustrato, han sido utilizadas como alternativas para el biosensado por su bajo costo de fabricación y fácil uso, debido a su capacidad de realce y confinamiento de la luz. En el proceso de biosensado, es común que un líquido fluya por encima del sustrato donde se encuentra la nanoestructura, por lo que existe un riesgo de desprendimiento de la misma. Por tanto, es deseable una incrustación parcial de la nanostructura en el sustrato, lo que modifica su respuesta óptica en condiciones ideales. En esta tesis, se estudia la respuesta óptica de una sola nanopartícula esférica de oro de 12.5 nm de radio, adecuada para metasuperficies de biosensado, cuando la nanoesfera se incrusta parcialmente en una sustrato plano de vidrio con una matriz de aire, e iluminada por una onda plana electromagnética con longitudes de onda en el rango óptico, considerando los dos estados de polarización así como diferentes ángulos de incidencia. Las secciones transversales de esparcimiento, absorción y extinción, así como el campo eléctrico inducido por la nanoesfera en los regímenes de campo cercano y lejano, se calculan con el método de elementos finitos y se comparan con las soluciones analíticas en dos casos límite: una nanoesfera embebida en una matriz infinita de aire, y en una matriz infinita de vidrio. Con base en los resultados numéricos obtenidos, se encontraron configuraciones óptimas para el biosensado considerando una metasuperficie desordenada conformada por nanoesferas de oro de 12.5 nm de radio en el régimen diluido.



\end{abstracts}
%\end{abstractlongs}


% ----------------------------------------------------------------------
