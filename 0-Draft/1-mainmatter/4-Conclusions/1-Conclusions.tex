% !TeX root = ../../protocolo.tex

In this thesis, the  optical properties of a spherical gold nanoparticle (AuNP) of radius $a = 12.5$~nm, partially embedded in an air matrix and in a glass substrate, was studied as a function of its embedding degree, characterized by the incrustation parameter $h/a$ ---with $h$ the position of the center of the AuNP relative to the planar air-glass interface---. By means of the Finite Element Method ---implemented in the commercial software COMSOL Multiphysics\texttrademark{} Ver. 5.4--- the absorption and scattering efficiencies, the radiation pattern and the spatial distribution of the induced electric field of the partially embedded 12.5~nm~AuNP were calculated when the AuNP was illuminated by an electromagnetic plane wave traveling at an oblique direction, with a defined polarization state; all numerical results were compared with the Mie-limiting cases calculated analytically, consisting in a 12.5~nm~AuNP embedded in an infinite matrix of air, and an infinite matrix of glass. From the preformed calculations, it was observed that the 12.5~nm~AuNP with partial embedding can be described by a mainly dipolar contribution, that its coupling with the incident light and the spatial distribution of the electric field enhancement on its surface can be tuned depending on the embedding of the AuNP and its illumination conditions, and that the optical response is maximized if the system is illuminated with an evanescent wave at an angle of incidence near the critical angle. More specifically, from the preformed  calculations the following can be concluded:

    \begin{itemize}
        \item \textbf{The optical response of a single partially embedded AuNP can be described by a mainly dipolar contribution.}\\
        The absorption and scattering efficiencies present only one global maximum in the visible spectrum, at which the spatial distribution of the electric field enhancement and its radiation pattern resemble that of an electric point dipole. This behavior can be extended to other materials of the matrix, the substrate and the nanosphere (of any size) as long as the scattering contribution to the extinction of light is small compared to the absorption contribution.
        %
        \item \textbf{There is a smooth transition between the two Mie-limiting cases as the nanosphere is partially embedded into the substrate.}\\
        The  wavelength of resonance of the absorption and scattering efficiencies of the partially embedded nanosphere is localized in between the two Mie-limiting cases, which consist in the nanosphere embedded in an infinite media (either the matrix or the substrate). Aditionally, the wavelength of resonance is redshifted from the resonance wavelength of the matrix Mie-limiting case to the resonance wavelength of the substrate Mie-limiting case, and this redshift is different for an $s$ or  for a $p$ polarized incident electric field.
        %
        \item \textbf{The optical response of the nanosphere resembles that of a supported (totally embedded) nanosphere if at most one eight of its volume is partially embedded in the substrate (matrix).}\\
        The supported and totally embedded nanosphere are the extreme cases of the partially embedded nanosphere when the sphere is tangential to the matrix-substrate interface. The absorption and scattering efficiencies of the partially embedded nanospheres, for both polarizations, are enhanced and redshifted in the same trend as the supported and totally embedded spheres as the angle of incidence of the incident light changes if at most one eight of the nanosphere crossed the interface.
        %
        \item \textbf{The optical properties of the partially embedded nanospheres are maximized if illuminated at an angle of incidence near the critical angle.}\\
        For any incrustation parameter and polarization state, the magnitude of the scattering and absorption efficiencies is enhanced ---for all wavelengths in the visble spectrum--- as the angle of incidence grows from zero to the critical angle, and they start to  diminish for angles of incidence above the critical angle. This behavior is due to the effect of an evanescent wave illuminating the system above the interface, whose penetration depth is maximum at the critical angle.
        %
        \item \textbf{The wavelength of resonance and the electric field spatial distribution of the partially embedded nanospheres for $s$ polarized illumination do not depend on the angle of incidence while they do for $p$ polarized illumination.}\\
        On the one hand, for $s$ polarization, the redshift of the resonance wavelength as the nanosphere is buried into the substrate, is  the same for all angles of incidence and the  electric field at the resonance wavelength is enhanced in two hotspots aligned parallel to the interface and on the surface of the nanosphere in the substrate side of the system. On the other hand, for $p$ polarization, the redshift of the resonance wavelength is different for each angle of incidence. For example, near the critical angle, the redshift is appreciable if more than half of the nanosphere is buried into the substrate, while for normal incidence the behavior is equivalent to the $s$ polarization case. On the spatial distribution, one hotspot is located in the matrix and other in the substrate, and their alignment is determined by the transmitted electric field; in particular, for angles above the critical angle, the hotspots are aligned perpendicular to the substrate.
    \end{itemize}

Finally, it can be concluded that the optical properties of a partially embedded spherical AuNP of radius 12.5 nm, with at most one eight of its volume buried into the substrate, is suited for interactions with elements in the matrix under internal illumination. If the system is illuminated with a $p$ polarized incident electromagnetic plane wave traveling at an angle  $\theta_i \gtrsim \theta_c$, the system is optimized to interact with its surroundings above the substrate since the optical response is maximized in the matrix. Therefore, partially embedded spherical AuNPs are strong candidates for meta-atoms conforming a disordered biosensing-aimed-metasurface.
