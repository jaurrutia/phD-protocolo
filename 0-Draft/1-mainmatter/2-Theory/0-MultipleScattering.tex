% !TeX root = ../../protocolo.tex

The study of wave propagation in complex media has long been a central topic in optics, acoustics, and condensed matter physics. In particular, understanding the optical response of ensembles of nanoparticles (NPs) has garnered significant attention due to its relevance in fields such as nanophotonics, plasmonics, and metamaterials. Two main theoretical frameworks are typically employed to model these systems: Multiple Scattering Theories (MSTs) and Effective Medium Theories (EMTs).

The MSTs explicitly accounts for the interactions between individual scatterers and provides an accurate description of wave propagation in structured and disordered media \cite{loiko_monolayers_1998,barrera1991optical,reyes2018analytical}. The Foldy-Lax equations, formulated in the mid-20th century, offer a rigorous way to model the self-consistent field scattered by a collection of particles. In contrast, effective medium theories aim to replace a heterogeneous system with a homogeneous medium having equivalent macroscopic optical properties, often utilizing approximations such as Maxwell-Garnett and Bruggeman models. While EMTs offer computational simplicity and analytical tractability, they are generally less accurate since, they inherently neglect multiple scattering effects, phase coherence, and near-field coupling between NPs. 

In the following sections, two main MSTs suited to describe the optical properties of ensenmbles of NPs constrained to a bidimensional 