The proposed project for the doctoral program at PCF arises from a theoretical-experimental collaboration with the Biophotonics Group—led by Prof. Rubén Ramos García—at the National Institute of Optics and Electronics (INAOE). This collaboration motivated my research topic during the PCF master's program and now aims to extend it by introducing increasingly realistic conditions in the theoretical description of metasurfaces. The general objective of the doctoral project is to develop semi-analytical extensions to the theories mentioned in Section \ref{s:Ante}—to describe metasurfaces in a realistic biosensing environment—and to compare them with experimental results, given that the Biophotonics Group has techniques for synthesizing disordered metasurfaces of spherical gold NPs and an experimental setup for real-time reflectance spectrum measurements \cite{cuanalo_sensitivity_2022}.

The goals for the first year of the doctoral program are as follows:
%
\begin{itemize}
	\item \textbf{Numerical implementation of the CSM considering size distribution}\\
	Implement the CSM expressions given by Eq. \eqref{eq:CSMpoly} for a size distribution determined by micrograph images. This code should account for the size correction of the dielectric function \cite{noguez_surface_2007,mendoza_herrera_determination_2014} of the spherical NP material to correctly compute the average size distribution.
	\item \textbf{Implementation of a parameter fitting method}\\
	Implement the gradient descent method \cite{barzilai_two-point_1988} to minimize the error between optical response calculations and experimental measurements. This tool should be applicable to any theoretical model considered.
	\item \textbf{Comparison and evaluation of measurement schemes with the CSM}\\
	Evaluate and compare the method for quantifying the optical response of the metasurface when changing the refractive index of the material in contact with it. This evaluation will be conducted by comparing the figure of merit \cite{estevez_trends_2014} for the following methods: minimum in the reflectance spectrum and phase shifts of ellipsometric parameters \cite{svedendahl_refractometric_2014,qiu_dual_2020}.
\end{itemize}
%
The first goal will enable the modeling of metasurfaces like those shown in Fig. \ref{fig:dist}, allowing the determination of optimal parameters for metasurface-based sensing applications, as well as setting a lower bound for the accuracy of the employed synthesis methods. The second goal will quantify the extent to which a model or theory describes the optical response of a metasurface while providing a tool for non-destructive metasurface characterization. The final goal focuses on analyzing the sensitivity of the metasurface and the feasibility of performing the corresponding measurements by quantifying the figure of merit of the two proposed schemes. These goals not only implement general methodologies and tools to evaluate the methods and approaches to be applied later, but also establish a path for a first publication on the characterization and performance of this type of metasurface in sensing, encompassing a theoretical-experimental analysis from synthesis to application.
